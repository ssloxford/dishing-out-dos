\section{Impact and Mitigations}\label{sec:impact}

Since this attack can be deployed from any device connected to the local network, large networks containing many untrusted users are at the greatest risk.
Examples may include maritime and aviation traffic, internet cafés, or large organizations.
Since the Starlink routers do not password protect the network by default, this is a serious concern.

There is also potential for remote ``drive-by'' attacks, provided the attacker can in some way cause a device on the same network as the dish to send HTTP requests.
This is because executing the attack only requires a few seconds of connection on the local network, and can cause outages on the order of minutes or hours.
The Cross-Origin Resource Sharing (CORS) policies of modern browsers prevent javascript from making unauthorized requests to external domains or addresses, so javascript-based attacks are unlikely unless legacy browsers are used~\cite{cors}.
However, the lack of encryption on the connection to ``http://my.starlink.com'' allows local attackers to hijack the DNS requests or respond with a malicious website to make the request instead.
Furthermore, the attacker could trick a user into executing a malicious executable or script, which could easily be used to make these requests.

We argue for the importance of securing these physical systems through satellite router configuration.
Even simple mitigations, such as adopting password protection, using local TLS certificates, and the use of HTTP Strict Transport Security~\cite{rfc6797} would prevent many such attack vectors.


Furthermore, a dedicated administrative wireless network would prevent drive-by attacks, since no device will be connected both to the administrative interface and the public internet.
This bears similarities to the ``guest network'' feature already provided, which provides an internet connection without access to the guest network.
