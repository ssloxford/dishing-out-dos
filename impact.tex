\section{Impact}\label{sec:impact}

This attack can have a significant impact -- as mentioned above, denial of service can be achieved by stowing the dish before sending the kill command, requiring the dish to by physically power-cycled before service is restored.
As long as the adversary remains on the local network, this attack can be repeated to cause continuous loss of service for users on the network.
Therefore, attackers that can maintain presence on the network will have the greatest impact.

Since the attack can be deployed from any device connected to the local network, large networks containing many untrusted users are at the greatest risk.
Such networks also suffer greater impact, as more devices are affected by network disruptions.
The impact is magnified when Starlink is the only source of internet access for that customer.
Examples may include maritime and aviation traffic, internet cafés, or large organisations.

There is also potential for remote attacks, provided the attacker can in some way cause a device on the same network as the dish to send HTTP requests.
The Cross-Origin Resource Sharing (CORS) policies of modern browsers prevent javascript from making unauthorized requests to external domains or addresses, so javascript-based attacks are unlikely unless legacy browsers are used~\cite{cors}.
However, the attacker could trick a user into executing a malicious executable or script, which could easily be used to make these requests.

Furthermore, in some cases ``drive-by'' attacks are possible -- if the network is not password protected, an attacker can connect and execute the attack while passing nearby.
Since the Starlink routers do not password protect the network by default, this is a serious concern.
Executing the attack only requires a few seconds of connection on the local network, and can cause outages on the order of minutes or hours.
This can be mitigated by securing the network with a password or, if an unprotected network is absolutely necessary, using the ``guest network'' mode provided by the router.
This adds an unprotected guest network which does not have access to the administrative interface.

Restoring service requires physical access to the terminal, so disruption will be increased where access is difficult or restricted.
Examples may include secured rooftop installations.


\subsection{Responsible Disclosure}\label{sec:responsible-disclosure}

This vulnerability has been reported to Starlink through their provided channels.
It has been triaged and reproduced by their security team, and the root cause was determined to be a bug in the gRPC server's handling of edge cases.
A fix has since been implemented in patch \textit{8c03f1b9-de75-404b-87fd-7986892cdacb.uterm.release} and deployed to Starlink user terminals in December 2022.
