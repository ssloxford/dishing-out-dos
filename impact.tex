\section{Impact}\label{sec:impact}

This attack can have a signficant impact -- since the state of the dish is frozen, an adversary can achieve persistent denial of service by first sending a command to stow the dish.
This will interrupt service until the dish can be physically power-cycled, which is not always trivial.
As long as the adversary remains on the local network, this attack can be repeated to cause continuous loss of service for users on the network.
Therefore, attackers that can maintain presence on the network will have the greatest impact.

\textbf{TODO drive-by attacks, potential for remote attacks on non-CORS browsers or through rogue executables}

Since the attack can be deployed from any device connected to the local network, large networks containing many untrusted users are at the greatest risk.
Such networks also suffer greater impact, as more devices are affected by network disruptions.
The impact is magnified when Starlink is the only source of internet access for that customer.
Examples may include maritime and aviation traffic, internet cafés, or large organisations.

Restoring service requires physical access to the terminal, so disruption will be increased where access is difficult or restricted.
Examples may include secured rooftop installations.

\subsection{Responsible Disclosure}\label{sec:responsible-disclosure}

This vulnerability has been reported to Starlink through their provided channels.
It has been triaged and reproduced by their security team, and the root cause was determined to be a bug in the gRPC server's handling of edge cases.
A fix for this problem will have been fully deployed by the time of this paper's publication.

\textbf{TODO something about how they suggest the ``Guest Network'' mode to isolate devices from being able to access the gRPC server.}

\textbf{TODO distinguish from discussion of design issues}

We would like to thank the Starlink responsible disclosure team for promptly confirming the issue and deploying a fix, and working with us to \textbf{TODO}.