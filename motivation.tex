\section{Motivation}\label{sec:motivation}

Traditionally, satellite internet usage has traded latency and bandwidth for global, reliable connectivity.
However, renewed private interest has resulted in satellite internet that can compete with terrestrial offerings, paving the way for mass market adoption.
As a result, satellite internet hardware is now designed to provide a good user experience, making configuration as easy as possible out of the box.
However, these constraints can sometimes come at the cost of security.

For example, the Starlink user terminal is configured by visiting the public website \texttt{my.starlink.com} from the same local network as the terminal.
The web page immediately shows the status of the dish, and allows the user to configure it.
To support this, the admin commands sent over the local network are not password protected.
This contrasts with admin pages as typically implemented on common consumer routers, which require connecting to a local IP address and password authentication.

It is well known that even password authenticated router administration pages are vulnerable to attackers on the local network; the password can be extracted from plaintext HTTP through network sniffing.
The Starlink user terminal is similarly vulnerable, since any device connected to the local network to explore and change the user terminal settings, which can include compromised devices.
This also allows drive-by javascript attacks, in which the user terminal can be configured from a public website, on legacy browsers that do not enforce the same-origin policy.

Additionally, certificates have not been issued to secure the traffic to the user terminal, meaning they are conducted in plain http.
\texttt{my.starlink.com} is therefore also served in plain http, since browsers do not permit resource sharing from a secure origin (https) to an insecure origin (http).
Although Starlink can prevent man-in-the-middle attacks by ensuring that \texttt{my.starlink.com} is only served through their encrypted satellite data link, this still opens the door for other attacks such as DNS hijacking and IP spoofing.

% As a result, exploits against the user terminal can be executed from the local network.


Test cite~\cite{roccaUnderstanding2021}.
