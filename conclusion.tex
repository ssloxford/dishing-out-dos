\section{Conclusion}\label{sec:conclusion}

We have seen that the Starlink router was vulnerable to a denial-of-service attack through the sending of malformed commands over the router's administrative interface.
Although this vulnerability has since been patched, it draws attention to weaknesses in the design of routers' administrative interfaces -- design choices intended to facilitate a more streamlined user experience lead to vulnerabilities which could be exploited by local attackers, or by exploiting victims' browsers.

We have explored the security challenges faced by the Starlink router in light of existing work on the security of routers more generally.
This has highlighted the challenges inherent in establishing a secure connection between the browser and router for administrative purposes, whilst maintaining user convenience.

Some technical improvements are required, but a significant factor in this is steering users into making well-informed choices to maximize security.
These choices include changing administrative passwords, updating TLS certificates, and making use of guest networks to reduce the risk of drive-by attacks.
Through good UX design, it is therefore possible to have a polished user experience without sacrificing security.

%We have explored these vulnerabilities, and show that it is possible to have a polished user experience without sacrificing security.
%Some technical improvements are required, but a significant factor in this is steering users into making well-informed choices to maximize security.
%These choices include changing administrative passwords, updating TLS certificates, and making use of guest networks to reduce the risk of drive-by attacks.
