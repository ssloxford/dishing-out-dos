\section{Discussion}\label{sec:discussion}

Yes

\begin{comment}
  Edd's notes:

Related work:

Zoom exploit, hitting services running on the local network

https://developer.mozilla.org/en-US/docs/Web/Security/Same-origin_policy
"Cross-origin writes are typically allowed"

https://bugzilla.mozilla.org/show_bug.cgi?id=629094
https://www.grepular.com/Abusing_HTTP_Status_Codes_to_Expose_Private_Information
HTTP status code abuse to extract information cross-origin

https://bugzilla.mozilla.org/show_bug.cgi?id=354493
Zoom videoconferencing problem

https://bugzilla.mozilla.org/show_bug.cgi?id=371598
Drive-by pharming through browsers


Security constraints:

Defeating drive-by javascript:
unencrypted passwords prevents silent drive-by javascript, potentially
websites can still brute force or phish the password from you
some sort of CSRF token from my.starlink.com?

Defeating spoofing my.starlink.com:
Vulnerable: because it's using plaintext http
Reason: you can't make a request to an insecure origin from a secure origin (need citation)
use https
passwords prevent silently changing the dish

Attackers on local network:

HSTS could be used with TLS to secure this
CSRF tokens

%Additionally, certificates have not been issued to secure the traffic to the user terminal, meaning they are conducted in plain http.
%%\texttt{my.starlink.com} is therefore also served in plain http, since browsers do not permit resource sharing from a secure origin (https) to an insecure origin (http).
%%Although Starlink can prevent man-in-the-middle attacks by ensuring that \texttt{my.starlink.com} is only served through their encrypted satellite data link, this still opens the door for other attacks such as DNS hijacking and IP spoofing.
%


\end{comment}
